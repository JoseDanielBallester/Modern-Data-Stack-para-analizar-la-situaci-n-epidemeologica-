\apendice{Plan de Proyecto Software}

\section{Introducción}

\section{Planificación temporal}
Antes de iniciar el proyecto, se decidió utilizar una metodología ágil para gestionarlo.
Se realizaron sprints de una o dos semanas. Al final de cada sprint se realizaba una reunión para hacer balance de la consecución de los objetivos marcados al inicio y proponer las tareas a realizar para el nuevo sprint.

Se utilizaron GitHub y ZenHub para la gestión de proyectos. El tablero proporcionado por ZenHub facilitó visualmente la organización del proyecto. Además, GitHub ayudó con la estimación de tareas, ya que tiene puntos de historia para ello.

\subsection{Sprint 1 - 16/03/2022-23/03/2022}
La primera semana se ha dedicado a elegir el gestor de proyectos, gestor de tareas, a configurar el repositorio del proyecto, a elegir el gestor de referencias, el editor de texto para la memorir y empezar la documentación, documentando este primer sprint.

\section{Estudio de viabilidad}

\subsection{Viabilidad económica}

\subsection{Viabilidad legal}


