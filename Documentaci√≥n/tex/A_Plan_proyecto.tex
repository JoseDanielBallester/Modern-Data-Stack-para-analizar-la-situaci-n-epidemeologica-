\apendice{Plan de Proyecto Software}

\section{Introducción}

\section{Planificación temporal}
Antes de iniciar el proyecto, se decidió utilizar una metodología ágil para gestionarlo.
Se realizaron sprints de una o dos semanas. Al final de cada sprint se realizaba una reunión para hacer balance de la consecución de los objetivos marcados al inicio y proponer las tareas a realizar para el nuevo sprint.

Se utilizaron GitHub y ZenHub para la gestión de proyectos. El tablero proporcionado por ZenHub facilitó visualmente la organización del proyecto. Además, GitHub ayudó con la estimación de tareas, ya que tiene puntos de historia para ello.

\subsection{Sprint 1 - 16/03/2022-23/03/2022}
La primera semana se ha dedicado a elegir el gestor de proyectos, gestor de tareas, a configurar el repositorio del proyecto, a elegir el gestor de referencias, el editor de texto para la memorir y empezar la documentación, documentando este primer sprint.
\subsection{Sprint 2 - 23/03/2022-06/04/2022}
El segundo sprint se ha dedicado a una formación básica en las principales herramientas que se va a usar en el TFG, snowflake, dbt y powerBI, y ádemas a hacer una primera versión básica de la arquitectura de datos que se va a construir para este TFG, utilizando las 3 herramientas principales antes mencionadas, la cual se irá ampliando en posteriores sprints. 
\subsection{Sprint 3 - 06/04/2022-20/04/2022}
El tercer sprint se ha dedicado a documentar el segundo sprint y a una formación avanzada en dbt ya que va a ser una herramienta que va a tener una parte importante de desarrollo de código. Durante este sprint se tenía planteado avanzar más, pero debido a que estuve enfermo gran parte del sprint se tuvo que transladar al siguiente.

\section{Estudio de viabilidad}



\subsection{Viabilidad económica}

\subsection{Viabilidad legal}


