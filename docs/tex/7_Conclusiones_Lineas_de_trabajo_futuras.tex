\capitulo{7}{Conclusiones y Líneas de trabajo futuras}

Conclusiones derivadas del desarrollo del proyecto.  

\section{Conclusiones}
Una vez finalizado el proyecto, se pueden considerar una serie de conclusiones basadas en su evolución:
\begin{itemize}
	\item Se ha podido comprobar que los objetivos generales se han cumplido de manera satisfactoria, ya que finalmente se ha logrado el desarrollo de una arquitectura de datos que recopila la información de las principales fuentes oficiales sobre el estado de la pandemia en España, mediante su extracción, carga y transformación en la nube. Y, además, de disponer toda la información recopilada almacenada en la nube, se ha facilitado al usuario la interacción con los datos.
	
	\item El proyecto ha requerido el uso de nuevos conocimientos sobre distintos conceptos y diversas herramientas.
	Respecto a los nuevos conceptos aprendidos puedo considerar que, en los que más he tenido que profundizar, son aquellos conceptos relacionados con el nuevo paradigma \emph{modern data stack}, entre los que se incluyen ELT y \emph{analytics engineer}. El aprendizaje de todos estos conceptos ha resultado muy útil debido al aumento de uso de esta nueva arquitectura que están teniendo las empresas españolas, gracias a las ventajas que ofrece.

	\item La evolución del proyecto ha sido controlada a través del desarrollo ágil de Scrum a partir de GitHub, lo que ha facilitado la organización y aceleración de la entrega final del proyecto. Gracias a la elaboración del proyecto en sprints se ha podido observar el incremento de valor y, a su vez corregir los errores localizados en sprints anteriores.
\end{itemize}

 \section{Líneas de trabajo futuras}
 Algunas de las líneas de trabajo futuras que se podrían considerar son:
\begin{itemize}
    \item Conseguir una automatización completa en la extracción de datos.
    \item Implementar algún análisis predictivo, mediante técnicas estadísticas de modelización, aprendizaje automático y minería de datos.
    \item Conseguir una versión premium de PowerBI para poder hacer público la URL del cuadro de mandos.
    \item Añadir más gráficos para la visualización de datos con nuevos KPIs en la herramienta empleada {\emph{Microsoft Power BI}}.
\end{itemize}
 


