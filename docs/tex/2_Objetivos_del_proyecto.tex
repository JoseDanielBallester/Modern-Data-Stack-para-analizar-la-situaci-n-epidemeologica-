\capitulo{2}{Objetivos del proyecto}

Este apartado explica de forma precisa y concisa cuales son los objetivos que se persiguen con la realización del proyecto. Se puede distinguir entre los objetivos marcados por los requisitos del software a construir y los objetivos de carácter técnico que plantea a la hora de llevar a la práctica el proyecto.


\section{Objetivos generales}\label{objetivos-generales}
\begin{itemize}
    \item Desarrollar una arquitectura de datos que recopile la información de las principales fuentes oficiales encargadas de publicar datos sobre el estado de la pandemia en España.
    \item Almacenar toda la información recopilada de forma organizada y accesible en la nube.
    \item Realizar la extración, carga y transformación de los datos automáticamente en la nube.
    \item Proporcionar un acceso visual a la información a través de representaciones gráficas.
    \item Facilitar al usuario la interacción con los datos mediante la aplicación de filtros para la selección de la información.
\end{itemize}

\section{Objetivos técnicos}\label{objetivos-tecnicos}
\begin{itemize}
    \item Usar Snowflake como data warehouse analítico en la nube para el almacenamiento de datos.
    \item Hacer uso de la herramienta DBT la transformación de los datos directamente en una datawarehouse en la nube.
    \item Usar la herramienta Fivetran para extración y carga de datos automática.
    \item Realizar la visualización de datos con la aplicación PowerBI.
    \item Utilizar la herramienta GitHub para realizar el control de versiones.
    \item Emplear la metodología ágil Scrum para la organización del proyecto (y el desarrollo del software).
    \item Usar la plataforma ZenHub para la gestión del proyecto.
\end{itemize}

\section{Objetivos personales}\label{objetivos-personales}
\begin{itemize}
    \item Realizar una aportación a la modernización del Business Intelligence y los sistemas OLAP
    \item Profundizar en el Modern Data Stack (MDS) mediante la utilización de tecnologías que ayudan a extraer, cargar, transformar y visualizar los datos
    \item Examinar metodologías, herramientas y aplicaciones actuales.
    \item Ampliar mis conocimientos en el modelado de datos en sistemas OLAP
    \item Profundizar en las transformaciones en la nube del nuevo método Extract-Load-Transform (ELT)
\end{itemize}