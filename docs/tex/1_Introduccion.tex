\capitulo{1}{Introducción}
La gran diversidad de fuentes de información en internet provoca desconfianza a la hora de consultar datos, ya que en muchas ocasiones la información es falsa o no está verificada.

Este trabajo tiene como objetivo brindar una solución visual e interactiva que facilite el acceso a información de la situación del Covid-19 en España.

Como respuesta a este problema se prevé desarrollar un cuadro de mandos que recopile los principales datos de las fuentes oficiales encargadas de publicar información sobre la situación actual del Covid-19 en el país, de forma que el usuario tenga una opción centralizada y fiable para consultar la información que desea y poder observar insights sobre ésta. 

Asimismo, otra novedad que aporta con respecto a las alternativas más clásicas, es la interacción del usuario con los datos, permitiendo una modificación dinámica de la visualización mediante el uso de filtros, de manera que permita obtener información específica y más detallada sobre un dato o un conjunto de datos.

Estos análisis son complicados con los sistemas transaccionales tradicionales (OLTP), porque la información está dispersa en muchas tablas y no están dirigidas al análisis. 

Debido a estos problemas, se ha introducido el uso de Business Intelligence y los sistemas analíticos (OLAP), donde la información se integra en un repositorio optimizado, para responder preguntas estratégicas. 

El Business Intelligence (BI) se conoce como la capacidad de transformar los datos en información y la información en conocimiento. Se logra a través del almacenamiento y procesamiento de grandes cantidades de datos para que la información pueda ser analizada.

Otro aspecto importante, es que en este TFG se pretende utilizar el Modern Data Stack, un nuevo paradigma a la hora de montar la arquitectura que da soporte al ciclo de vida de los datos, gracias a las ventajas que tiene montar esta estructura en la nube.

El término “data stack” se origina en “technology stack”, la combinación de diferentes tecnologías, que los ingenieros de software combinan para crear productos y servicios. 

Los data stack están diseñados específicamente para respaldar el almacenamiento, la administración y el acceso a los datos y se usan para aprovechar los datos en la toma de decisiones estratégicas. 

El Modern Data Stack (MDS) consiste en un conjunto flexible de tecnologías que ayudan a almacenar, administrar y aprender de los datos.

La nube ha transformado la industria del software y la forma en que las empresas construyen sus productos. Hoy, podemos configurar un technology stack en una fracción del tiempo menor y de forma más económica. Y no sorprende que estas transformaciones hayan allanado el camino para el Modern Data Stack (MDS) y un cambio de paradigma en este tipo de soluciones. 

\section{Estructura de la memoria}\label{estructura-de-la-memoria}
La memoria está formada por la siguiente estructura:
\begin{itemize}
\tightlist
\item
  \textbf{Introducción:} concisa presentación del problema a solucionar junto con la resolución sugerida. Se incluye la estructura de la memoria y los materiales adjuntos.
\item
  \textbf{Objetivos del proyecto:} definición de los objetivos que
  pretende alcanzar el proyecto.
\item
  \textbf{Conceptos teóricos:} breve desarrollo de los conceptos
  teóricos fundamentales para la correcta interpretación del resultado planteado.
\item
  \textbf{Técnicas y herramientas:} exposición de métodos y
  herramientas empleadas para organización y desarrollo del proyecto.
\item
  \textbf{Aspectos relevantes del desarrollo:} muestra de fases
  relevantes a destacar durante el transcurso de la elaboración del proyecto.
\item
  \textbf{Trabajos relacionados:} estado del arte en el campo de las arquitecturas de datos para el análisis de datos mediante reporting.
\item
  \textbf{Conclusiones y líneas de trabajo futuras:} conclusiones
  extraídas de los resultados obtenidos a partir de la realización del proyecto y opciones de mejora o
  ampliación de la resolución planteada.
\end{itemize}

\section{Materiales adjuntos}\label{materiales-adjuntos}
\begin{itemize}	
	\item \textbf{Anexos:}
	\begin{itemize}
	\tightlist
	\item
  		\textbf{Plan del proyecto software:} organización y potencialidad del proyecto.
	\item
  		\textbf{Especificación de requisitos del software:} se incluye la
  		fase de análisis; los objetivos generales, el catálogo de requisitos
  		del sistema y la especificación de requisitos funcionales y no
  		funcionales.
	\item
  		\textbf{Especificación de diseño:} se expone la fase de diseño; el
  		ámbito del software, el diseño de datos, el diseño procedimental y el
  		diseño arquitectónico.
	\item
  		\textbf{Manual del programador:} comprende los aspectos más relevantes
  		relacionados con el código fuente (estructura, compilación,
  		instalación, ejecución, pruebas, etc.).
	\item
  		\textbf{Manual de usuario:} guía de usuario para el buen uso y empleo adecuado de la aplicación.
	\end{itemize}
	
	\item \textbf{Repositorio Github:} contiene la memoria y los anexos, además del archivo del cuadro de mandos, los accesos los sercivios y herramientas cloud utlilizados y los scripts de dbt, Snowflake y Phyton del proyecto.
	
\end{itemize}
